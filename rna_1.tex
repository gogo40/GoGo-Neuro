\documentclass[10pt,journal,letterpaper,compsoc]{IEEEtran}
\usepackage{graphicx}
\usepackage{hyperref}

\usepackage{multirow}

\usepackage[T1]{fontenc}
\usepackage{amsmath}
\usepackage{amsfonts}
\usepackage[utf8]{inputenc}

\makeatletter
%%\usepackage{babel}
\usepackage[brazilian]{babel}
\makeatother


\newcommand{\inte}[4]{
	\displaystyle \int_{#1}^{#2} #3 d#4
}

\newcommand{\fra}[2]{
	\displaystyle \frac{{#1}}{{#2}} 
}
\newcommand{\frap}[2]{
	\left( \displaystyle \frac{#1}{#2} \right)
}

\newcommand{\su}[2]{
	\displaystyle \sum_{#1}^{#2}
}

\newcommand{\df}[2]{
	\fra{d \left[ #1 \right]}{d#2}
}

\newcommand{\dfa}[2]{
	\fra{d #1 }{d#2}
}

\newcommand{\dpa}[2]{
	\fra{\partial #1 }{\partial #2}
}

\title{$9^o$ Trabalho de Redes Neurais}
\author{Péricles Lopes Machado}

\begin{document}

\maketitle

\begin{abstract}

Neste trabalho, algumas rede neurais RBF são treinadas para simular a equação de recorrência:
\begin{equation}
x_k = x_{k-1} + \fra{a x_{k-s} } { 1 + x_{i-s} ^ c} - b x_{k-1} + 0.1 N ,
\label{eq.1}
\end{equation}
onde $N$ é uma variável aleatória
de distribuição normal com $\mu = 0$ e $\sigma^2 = 1$, $a = 0.2$,
$b = 0.1$, $c = 10$ e $s = 17$. Além disso, é realizado um estudo
para se tentar obter a melhor escolha
de parâmetros anteriores da série para minimizar o erro quadrático associado a rede.

\end{abstract}

\section{As rede neurais}

As redes neurais possuem 7 neurônios, cujos centros associados foram escolhidos de forma aleatória
dentro da base de dados filtrada contendo 3 mil amostras da série $x_k$. A distância $D$,
utilizada para avaliar a distância entre uma entrada e o centro associado ao neurônio, foi a euclidiana.
A função de ativação utilizada foi $u_i(\vec{x}) =  e^{-\beta  D(\vec{x} - \vec{c_i})}$, onde $c_i$ é
o centro associado ao neurônio $i$ e o parâmetro livre $\beta$, nos testes realizados, foi $0,1$.

\section{O tipos de entrada da rede}
Foram definidos cinco tipos de entradas para a rede:

\begin{itemize}
\item $I_1 = \{x(k-1)\}$
\item $I_2 = \{x(k-1), x(k-2)\}$
\item $I_3 = \{x(k-1), ... , x(k-3)\}$
\item $I_4 = \{x(k-1), ... , x(k-4)\}$
\item $I_5 = \{x(k-a), x(k-b)\}$
\end{itemize}

No caso do $I_5$, tenta-se definir uma escolha de $a$ e $b$ que minimiza o erro quadrático.

Para cada tipo de entrada foi gerado um banco de dados de treino e teste utilizando-se o mesmo algoritmo
de filtragem apresentado nos trabalhos 7 e 8. O tipo de entrada, a partir de agora identificará a rede
neural associada.

\section{O treinamento \label{treina}}

Como no trabalho 8, o método de mínimos quadrados foi utilizado para o ajuste dos pesos de cada rede,
usando o mesmo {\it database} mas variando a configuração da entrada conforme foi descrito na seção anterior.

No caso da rede de tipo $I_5$, foi adotada a seguinte heurística:

\begin{itemize}
\item São geradas $NR$ séries $(x_i)_k$, com $i = 1, 2, ... ,M$ e $k = 1, 2, ... , NR$,
como definida na Eq. (\ref{eq.1}).
\item Para cada sequência $(x_i)_k$, define-se uma variável 
$dP_k = \fra{1}{N_{picos}}\su{n=2}{N_{picos}}|p_n - p_{n-1}|$, onde $p_n$ é
a posição do n-ésimo pico da série $(x_i)_k$.
\item Calcula-se o valor de $P = \fra{1}{NR} \su{n = 1} {NR}dP_n$.
\end{itemize}

A variável aleatória $dP_k$ é uma estimativa para o período da série $(x_i)_k$ que utiliza a distância 
entre picos da série para se obter o valor do período. A posição de um pico na série $(x_i)_k$ é um
valor $p$ tal que $(x_p)_k > (x_{p-1})_k$ e $(x_p)_k > (x_{p+1})_k$.

Neste trabalho foram geradas $NR = 5000$ séries com $M = 5000$ elementos cada uma. A distribuição de
$dP_k$ pode ser observada na Fig. \ref{fig.picos}.


\begin{figure}[!htb]
     \centering
     \includegraphics[scale=0.5]{./t9/picos.png}
     \caption{Distribuição de $dP$ em 5000 séries.}
     \label{fig.picos}
\end{figure}

A média da váriavel $dP$ nas 5000 séries geradas foi de $17,8154$, a variância foi de $0,5028$ e
o desvio padrão foi de $0,709$. Isso indica que o período de uma série $(x_i)_k$ tem grande probabilidade
de está contido no intervalo $[17, 18]$. Por isso, neste trabalho foram testados duas configurações de $I_5$,
uma com $a=1$ e $b=17$ e outra com $a=1$ e $b=18$.


\section{Resultados}

Foi utilizado o mesmo algoritmo de treinamento do trabalho 8, o mínimos quadrados, para ajustar
os pesos de cada tipo de rede. A seguir, temos o erro obtido para cada configuração:

\begin{table}[htb]
\caption{Erro quadrático para cada tipo de rede treinada para o mesmo conjunto de teste
do trabalho 7 e 8.}
\label{table.I}
\centering
\small

\begin{tabular}{c|c}

\hline
Tipo de rede & Erro quadrático\\
\hline
$I_1$ & $1.545211$\\
$I_2$ & $0.3560334$\\
$I_3$ & $0.2470221$\\
$I_4$ & $0.5016882$\\
$I_5$ com $a=1$ e $b=18$ & $0.2042160$\\
$I_5$ com $a=1$ e $b=17$ & $0.099801$ \\
\hline

\end{tabular}

\end{table}

Os testes foram realizados usando uma série diferente da usada no treinamento.


\begin{figure}[!htb]
     \centering
     \includegraphics[scale=0.5]{./t9/f1t.png}
     \caption{Saída gerada pela rede neural $I_1$ para o conjunto de teste do trabalho 7 e 8.}
     \label{fig.teste.1}
\end{figure}



\begin{figure}[!htb]
     \centering
     \includegraphics[scale=0.5]{./t9/f2t.png}
     \caption{Saída gerada pela rede neural $I_2$ para o conjunto de teste do trabalho 7 e 8.}
     \label{fig.teste.2}
\end{figure}



\begin{figure}[!htb]
     \centering
     \includegraphics[scale=0.5]{./t9/f3t.png}
     \caption{Saída gerada pela rede neural $I_3$ para o conjunto de teste do trabalho 7 e 8.}
     \label{fig.teste.3}
\end{figure}


\begin{figure}[!htb]
     \centering
     \includegraphics[scale=0.5]{./t9/f4t.png}
     \caption{Saída gerada pela rede neural $I_4$ para o conjunto de teste do trabalho 7 e 8.}
     \label{fig.teste.4}
\end{figure}



\begin{figure}[!htb]
     \centering
     \includegraphics[scale=0.5]{./t9/f5t.png}
     \caption{Saída gerada pela rede neural $I_5$ com $a=1$ e $b=18$ para o conjunto de teste do trabalho 7 e 8.}
     \label{fig.teste.5}
\end{figure}

\begin{figure}[!htb]
     \centering
     \includegraphics[scale=0.5]{./t8/saidaRBFt.png}
     \caption{Saída gerada pela rede neural $I_5$ com $a=1$ e $b=17$ para o conjunto de teste do trabalho 7 e 8.}
     \label{fig.teste.6}
\end{figure}

As Fig. \ref{fig.teste.1} a \ref{fig.teste.6} mostram as aproximações geradas pelas redes neurais deste trabalho.

\section{Conclusão}
Conforme pode-se observar na Tabela \ref{table.I}, as redes neurais que utilizam a $17^a$ ou a
$18^a$ amostra anterior foram
as que obtiveram o menor erro quadrático. Isso mostra que a heurística apresentada na seção  \ref{treina}
nos fornece uma boa estimativa para o período de uma série temporal e a utilização de um atraso igual
a esse período permite obtermos uma rede neural com baixo erro quadrático.

Além disso, a utilização da rede RBF permitiu reduzir significativamente o tempo de treinamento. Basicamente,
utilizando-se esse tipo de rede as principais tarefas se concentram no pré-processamento da base de dados e
na definição da entrada mais apropriada. Neste trabalho, não houve grandes problemas para se determinar
os centros associados a cada neurônio, mas sabe-se que esse pode ser um problema em situações mais gerais.
A heurística mostrada no trabalho 7 do autor permitu uma bom método para se definir os centros dos neurônios.

\end{document}

